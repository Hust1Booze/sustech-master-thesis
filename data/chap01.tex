% !TeX root = ../sustechthesis-example.tex

\chapter{研究背景和意义}

混合整数规划(Mixed Integer Programming,MIP)是运筹学和组合优化中的一个重要分支,许多组合优化问题从理论上都可以归结为MIP问题。近年来,随着 MIP 相关研究的深入和商业求解器的不断发展,MIP在交通运输\cite{henderson2006handbooks},生产计划\cite{pochet2006production}以及能源系统\cite{wirtz2021design}等领域得到了广泛应用。


\section{研究背景}

随着计算机技术的飞速发展,求解大规模优化问题的能力显著提升,MIP模型通过其强大的建模能力能有效应对现实中的复杂决策问题,为多个行业的优化提供了有力支持。在工业运筹优化中,常见问题如资源调度、生产安排、路径优化以及研发设计等,通常具备复杂的约束条件和多样化的目标函数,难以通过简单的方法解决。通过 MIP 建模,这些问题可以被形式化为数学模型,并利用现代优化技术进行高效求解。在交通运输领域,MIP 被广泛应用于车辆路径问题(Vehicle Routing Problem),用于优化运输路线,降低成本并提高效率;在生产计划中,MIP 可以帮助企业合理制定生产排程,优化资源配置,最大化产出并减少延误;在能源系统中,MIP 可以用于电网优化、可再生能源调度等,促进资源的高效利用和可持续发展。

随着求解问题规模的不断扩大,快速且高效的求解MIP问题已成为企业和科研界的共同需求。为应对这一挑战,基于分支定界法\cite{land2010automatic}(Branch and Bound,B\&B)的现代MIP求解器应运而生, 包括SCIP\cite{achterberg2009scip}、CPLEX\cite{cplex}、Gurobi\cite{Gurobi}等。这些求解器结合了多种优化技术,通过高效的算法和策略显著提升了求解速度和准确性,使得求解大规模 MIP 问题成为可能。

尽管现代求解器的性能已大幅提升,但仍然面临诸多挑战。作为 NP 难问题,MIP 与其他组合优化问题一样,通常需要依赖专家设计的启发式算法,这一过程耗时且依赖于丰富的经验。同时,求解过程对计算资源的需求较高,即便在使用先进的 MIP 求解器时,也需要调整数千个与决策密切相关的参数,而默认参数往往无法发挥出求解器的最佳性能,为了进一步提高 MIP 的求解效率,还需要针对问题特性进行手动调整。此外,求解器在处理复杂实际问题时,可能会陷入局部最优解,从而无法找到全局最优解。


在此背景下,如何提高 MIP 求解的效率和准确性,成为研究中的关键问题。自动化算法设计的研究逐渐受到重视,自动化设计不仅能够减少对专家知识的依赖,还能根据不同问题类型自适应地调整求解算法及超参数,从而显著提升求解器的性能。

近年来,机器学习和人工智能技术的迅速发展为 MIP 的自动化设计带来了全新的机遇。借助数据驱动的方法,可以深入分析历史求解数据,识别影响求解性能的关键因素,并基于这些信息进行算法的自动优化。例如,一些研究\cite{gasse2019exact,labassi2022learning,parsonson2023reinforcement}已经利用模仿学习和强化学习,训练神经网络来自动设计算法,替代手工编写的启发式方法,并通过智能代理动态调整算法参数,从而显著提升求解效率。

此外,将领域知识与自动化技术相结合也成为一大趋势。通过融合专家经验与数据分析,可以构建更为精准的模型,在特定应用场景下实现更优的求解效果。这种跨学科的研究方法不仅推动了 MIP 求解整体水平的提升,还为相关领域的研究开辟了新的方向。

\section{研究意义}

混合整数线性规划问题的求解难点主要在于其复杂性来自于离散变量与线性约束之间的组合爆炸式增长。针对该类问题,目前主要的求解策略可以分为两类:一类是基于启发式的近似方法,强调在有限时间内快速获得高质量的可行解;另一类是基于分支定界框架的精确方法,致力于在理论上保证找到全局最优解。

其中,元启发式算法是一类通过模拟自然界或社会系统行为的搜索策略,用于快速获取高质量的可行解,尤其适用于对解的最优性要求不高但求解时间受限的问题场景,例如大规模调度、实时决策支持系统等。这类方法无法提供最优性保证,但其灵活性强、适应性高,在许多工业实际中表现出良好的效果。

与之相对,分支定界算法属于精确算法范畴,能够在满足精度容忍度的条件下找到问题的全局最优解,因此在对最优性要求严格的应用中不可或缺,例如金融优化、航空排班、供应链设计等。但其求解效率高度依赖于搜索策略和问题结构,面对大规模复杂实例时可能遭遇维度灾难。

在MILP的求解领域,启发式方法和分支定界法各有优势与局限,二者形成互补关系:启发式方法强调计算效率与可扩展性,而精确方法关注解的质量与收敛性。围绕这两类方法的算法自动设计研究,能够进一步提升其适应性和性能,对于不同类型的MIP问题提供更具针对性的解决策略。

本研究旨在探讨混合整数优化算法的自动设计方法,解决当前 MIP 求解中面临的挑战。通过分析现有的两种基本求解方法,本研究将探讨所提出自动算法设计方法的可行性与有效性,并通过实证研究验证提出的方法在不同类型问题上的应用效果。希望通过这一研究,不仅能够推动 MIP 技术的进一步发展,还能为实际应用提供有价值的参考。

本研究的意义在于推动混合整数规划领域的自动化发展,提升求解效率,减少对专家知识的依赖,进而扩展 MIP 在各个行业的应用潜力。通过探索自动化算法设计,我们期望能够为复杂优化问题提供更加高效和灵活的解决方案,并为应对现实中的复杂决策问题提供实践基础。
