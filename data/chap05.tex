% !TeX root = ../sustechthesis-example.tex

\chapter{已完成的研究工作与进度安排}

\section{已完成工作}

\subsection{混合整数规划问题的元启发式算法的自动设计研究}

在本研究中,我们已完成对元启发式算法的自动化设计,并通过实验验证了该方法的有效性。相关成果已以导师为第一作者、本人为第二作者发表在《IEEE Transactions on Evolutionary Computation》上,论文题目为《Automated Metaheuristic Algorithm Design With Autoregressive Learning》。本研究提出的基于自回归学习(Autoregressive Learning)的自动设计方法,能够有效地设计与调优元启发式算法,使其适应不同的优化问题。通过对多种优化任务的实验评估,验证了该方法在多种问题上相较于传统启发式算法的优越性。此外,提出的持续学习机制用于保存以往的设计经验,实验结果表明,该方法相较于从零开始设计算法更加高效。总体上,我们已实现了研究设定的主要目标。

\subsection{混合整数规划问题的分支定界算法中节点选择策略自动设计研究}  
本研究围绕分支定界算法中节点选择策略的自动化设计展开,提出基于因果建模的节点选择方法,旨在通过端到端学习机制替代传统启发式规则,提升算法在复杂MILP问题中的求解效率与鲁棒性。相关成果已整理完成,并以本人一作、导师通讯投稿至《INFORMS Journal on Computing》,论文题目为《Learn to Select Node in Branch-and-Bound with Causality Modeling》。  本研究通过对比学习显式建模节点间的最优性传递性,使模型能够识别包含最优解的节点路径,而非依赖传统方法中的浅层统计相关性。  本方法在节点选择质量与求解效率上均优于现有深度学习方法(如SVM、RankNet、L2C)及启发式策略(如Estimate);在FCMCNF和MAXSAT两类问题中,所提方法的求解时间与分支定界树规模均显著优于SCIP求解器。

该研究通过将因果推理机制引入分支定界算法,为MILP求解提供了兼具效率、泛化性与可解释性的新范式,相关成果已进入同行评审阶段,后续将进一步完善实验分析与理论验证。  


\subsection{混合整数规划问题的分支定界算法的基础模型研究}

本研究已完成基础模型的设计与数据准备工作,具体进展如下:  

1. Decision Transformer模型构建:

(1) 完成基于Transformer架构的决策模型搭建,包括共享编码器、任务特定输出头(节点选择头与变量选择头)及动态任务权重调整模块。

(2)实现因果掩码机制与位置编码策略,确保模型学习时序一致的搜索路径。  

(3)采用DeepSpeed框架进行分布式训练,支持长序列显存优化。 

2.数据收集与预处理:

(1)从SCIP求解日志中提取GISP、FCMCNF、MAXSAT等10,000个MILP实例的完整搜索轨迹,涵盖状态、节点、奖励、分支变量等关键信息。

(2)编写序列生成相关代码: 使用图神经网络与MLP嵌入模块,将原始问题特征、节点特征及分支操作转化为混合类型序列(见表~\ref{tab:token_types}与式~\ref{eq:bnb_seq})。

在模型的初步实验中,我们验证了Decision Transformer的训练流程与基础性能。在预训练阶段,模型已成功学习到节点选择与变量选择的基本决策模式,初步验证了端到端建模的可行性。



\section{进度安排}

未来的研究工作将围绕以下两个核心阶段展开:

\subsection{全面预训练实验}  
本阶段将系统开展决策Transformer模型的预训练工作,重点包括:  
\begin{itemize}  
    \item \textbf{大规模预训练验证}:  
    在GISP、FCMCNF、MAXSAT等10,000个MILP实例上进行端到端预训练,评估模型在节点数、求解时间等指标上的表现。  
    \item \textbf{超参数敏感性分析}:  
    系统测试学习率、批量大小、等超参数对微调结果的影响,建立参数调优的指导原则。在FCMCNF、MAXSAT、GISP等基准测试集上进行消融实验,量化各模块(如因果掩码、位置编码)对性能的贡献。  
\end{itemize}  

\subsection{微调实验设计与优化}  
在预训练模型基础上,针对具体MIP问题进行微调实验,重点包括:  
\begin{itemize}  
    \item \textbf{微调策略探索}:  
    对比不同微调方案:① 冻结预训练编码器,仅训练任务头网络;② 逐步解冻模型层以保留通用特征;③ 全量微调以适应特定问题结构。  
    设计动态任务权重调整机制($\alpha, \beta$),根据验证集表现自动优化多任务协同效果。  

    \item \textbf{跨领域迁移验证}:  
     在MIPLIB、TSP等异构问题上测试微调后模型的泛化能力,分析其在未参与训练问题类型中的表现。  
    对比SCIP验证模型在求解效率与质量上的优势。  
\end{itemize}  

通过上述两个阶段的系统实验,本研究将完成从预训练到微调的全流程优化,为混合整数规划问题的自动化求解提供可复用、高性能的基础模型范式。  

具体任务的时间安排如下:

\begin{table}
  \centering
  \caption{研究进度时间安排表}
  \begin{tabular}{p{3cm} p{5cm} p{5cm}}
    \toprule
    时间(年-月)          & 主要任务          & 备注 \\
    \midrule
    2024-10              & 开题报告准备与开题答辩          & 与导师讨论,确定方向与方法 \\
    2024-11 \textasciitilde 2025-04     & 论文相关文献阅读,进行数据收集 & 调研图数据结构的数据增强方法,收集增强数据 \\
    2025-05 \textasciitilde 2025-08    & 实验方案完善,核心实验开展          & 搭建对比学习框架,验证不同的数据增强方法 \\
    2025-06               & 中期报告准备与中期考核            & 汇报进展,分析当前成果与问题 \\
    2025-10 \textasciitilde 2025-12    & 深入实验及数据分析               & 在下游任务上微调,实验不同的微调策略 \\
    2026-01 \textasciitilde 2026-03    & 论文撰写初稿                   & 梳理研究成果与实验数据,写初稿 \\
    2026-04               & 论文修改与完善                  & 与导师讨论,进行内容调整 \\
    2026-05              & 论文查重与格式调整                & 确保符合学校的格式与查重要求 \\
    2026-06              & 学位论文定稿与提交                & 提交最终版本,准备答辩 \\
    2026-07              & 答辩准备与正式答辩                & 准备答辩PPT和答辩内容 \\
    2026-07             & 完成学位申请与相关材料提交          & 最终提交学校要求的所有材料 \\
    \bottomrule
  \end{tabular}
  \label{tab:time-schedule}
\end{table}


